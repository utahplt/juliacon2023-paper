% JuliaCon proceedings template
\documentclass{juliacon}
\setcounter{page}{1}

\begin{document}

% **************GENERATED FILE, DO NOT EDIT**************

\title{My JuliaCon proceeding}

\author[1]{Taylor Allred}
\author[1]{Xinyi Li}
\author[1]{Ashton Wiersdorf}
\author[1]{Ben Greenman}
\author[1]{Ganesh Gopalakrishnan}
\affil[1]{University of Utah}

\keywords{Julia, Optimization, Game theory, Compiler}

\hypersetup{
pdftitle = {My JuliaCon proceeding},
pdfsubject = {JuliaCon 2019 Proceedings},
pdfauthor = {1st author, 2nd author, 3rd author},
pdfkeywords = {Julia, Optimization, Game theory, Compiler},
}


\maketitle

\begin{abstract}
  Reliable numerical computations are central to HPC and ML.
  We present FlowFPX: a Julia-based tool for tracking the onset and flow of IEEE Floating-Point exceptions that signal numerical defects.
  FlowFPX's design exploits Julia's operator overloading to trace exception flows and even inject exceptions to accelerate testing.
  We present intuitive visualizations of summarized exception flows including how they are generated, propagated and killed, thus helping with debugging and repair.
\end{abstract}

\section{Introduction}

\section{To Kill a Floating Point}
% If anyone has a better "kill" pun/reference than "To Kill a Mockingbird", I'm all ears.

% I think we will definitely need a clear, detailed section on how NaNs crop up and how they can be killed.

\section{FloatTracker Internals}

FloatTracker works by leveraging Julia's type-based method dispatch system to capture everything from calls to the standard library down to basic arithmetic operations.
We do this by creating custom types \texttt{TrackedFloat16}, \texttt{TrackedFloat32}, and \texttt{TrackedFloat64} that wrap their corresponding \texttt{Float16}, \texttt{Float32} etc. counterparts.
We then defined methods for every function in the \texttt{Base} module, including new implementations for operators like \texttt{+} and \texttt{*}.
These new methods had to be defined for every combination of tracked/untracked arguments;
fortunately Julia's metaprogramming capabilities make this relatively simple.

The surrogate methods in place, every operation on a TrackedFloat can be intercepted.
We have three primary goals when intercepting a float:

\begin{enumerate}
  \item Monitor the propagation of NaNs through the program
  \item Catch instances of NaN kills
  \item Optionally \emph{inject} NaNs to fuzz the program under scrutiny against NaN-kills
\end{enumerate}

% Good things to put in this section:
% - start with the max example, perhaps?
% - trace examples
% - CSTGs
%
% Don't forget to talk about trace recording

\section{Case Studies}

To illustrate what FloatTracker is capable of, we applied our tool to three different Julia libraries to help us track down strange behavior:

\begin{description}
\item[Surprise NaNs] We found that otherwise innocuous values for a time step parameter in the \texttt{ShallowWaters} library can produce NaNs in the result.
  We used FloatTracker to track down where the NaNs arose.
\item[NaN Fuzzing] We injected NaNs in a N-Body simulation and were able to find a bug in the widely-used \texttt{OrdinaryDiffEq} library.
\item[NaN Kills] We observed issues with NaN kills inside of the \texttt{Finch} library.
  % FIXME: more needed here
\end{description}

All three libraries are available from the JuliaHub package archive.
The \texttt{OrdinaryDiffEq} library in particular enjoys widespread use in the scientific computing community.

\subsection{Surprise NaNs: ShallowWaters}

\subsection{NaN Fuzzing: OrdinaryDiffEq}

The second thing we evaluated was an N-Body simulation library.
We didn't find any sensible configuration of input parameters that lead to NaN kills like in the ShallowWaters library.
We turned to the NaN injection capabilities of FloatTracker to fuzz the library to find any lurking bugs.
% FIXME: add hyper-ref to § 4.1: ShallowWaters

We configured FloatTracker to inject a single NaN to see if we could find any NaN kills.
On some runs we noticed that we would get a kill and the program would warn about a NaN and prepare to exit.
Contrary to the error message's claim, however, the program went into an infinite loop.

The logs lead us to a routine in the widely-used \texttt{OrdinaryDiffEq} library.
During initialization, a NaN got injected during the execution of \texttt{-} between two \texttt{TrackedFloat}s:

\begin{verbatim}
…
- at FloatTracker/src/TrackedFloat.jl:89
#__init#628 at OrdinaryDiffEq/src/solve.jl:106
…
\end{verbatim}

That injection point is here on third line of this snippet, which corresponds to line 106 from the \texttt{solve.jl} file in the \texttt{OrdinaryDiffEq} library:

\begin{lstlisting}[language = Julia]
tType = eltype(prob.tspan)
tspan = prob.tspan
tdir = sign(tspan[end] - tspan[1])

t = tspan[1]
\end{lstlisting}

\texttt{sign} correctly propagates NaNs, so \texttt{tdir} now contains a NaN.

We \emph{could} trace this variable through the code, but that would be a lot of grunt work.
Instead, we can look at the NaN kill logs to get a clue as to where we should look next:

\begin{verbatim}
…
< at FloatTracker/src/TrackedFloat.jl:193
solve! at OrdinaryDiffEq/src/solve.jl:515
…
\end{verbatim}

We get a very large kill file from this run, and the two lines from above show up repeatedly.
Thus, this is a good candidate location for where the cause of the loop is.

The relevant part of \texttt{solve.jl} looks like this:

% Note for Ashton:
% file here: ~/Research/ode_debug/dev/OrdinaryDiffEq/src/solve.jl
% see line 514

\begin{lstlisting}[language = Julia]
while !isempty(time_stops)
  while tdir * t < first(time_stops)
    # do integration work
  end
  pop_if_work_done(time_stops)
end
\end{lstlisting}

The problem here is on line 2 when \texttt{tdir} is NaN: multiplication propagates NaNs, and comparison kills it, so the condition on the inner \texttt{while} loop is \emph{always} false.
No work got done in the inner loop, and so the conditional \texttt{pop} routine never reduced the size of the \texttt{time\_stops} vector.

Our kill logs led us right to this line; this is a perfect example of how NaN kills can influence control flow to go awry.
In this case, the problem was apparent besides what our logs told us and it manifested as an infinite loop.
More dangerous cases can occur when the influence of a bad comparison is not as readily observable.

\subsection{NaN Kills: Finch}

% Advection examples go here, right?

\section{Evaluation}

\section{Discussion}

\section{Related Work}

\section{Acknowledgments}

% **************GENERATED FILE, DO NOT EDIT**************

\bibliographystyle{juliacon}
\bibliography{ref.bib}


\end{document}

% Below is some Emacs-specific stuff. If you're using Emacs, you should try
% using the new jinx package for spell-checking!

% Local Variables:
% jinx-local-words: "CSTGs JuliaHub OrdinaryDiffEq ShallowWaters TrackedFloat"
% End:
